\documentclass{resume}
\usepackage{zh_CN-Adobefonts_external} 
\usepackage{linespacing_fix}
\usepackage{cite}
\usepackage{hyperref}
\hypersetup{
    colorlinks=true,
    linkcolor=cyan,
    filecolor=magenta,      
    urlcolor=blue,
}

\begin{document}
\pagenumbering{gobble}

%***"%"后面的所有内容是注释而非代码,不会输出到最后的PDF中
%***使用本模板,只需要参照输出的PDF,在本文档的相应位置做简单替换即可
%***修改之后,输出更新后的PDF,只需要点击Overleaf中的“Recompile”按钮即可

%在大括号内填写其他信息,最多填写4个,但是如果选择不填信息,
%那么大括号必须空着不写,而不能删除大括号。
%\otherInfo后面的四个大括号里的所有信息都会在一行输出
%如果想要写两行,那就用两次这个指令(\otherInfo{}{}{}{})即可

\sepspace

%***********个人信息**************
\MyName{XXX}
\sepspace
%\SimpleEntry{哈尔滨工业大学}
\SimpleEntry{X\hspace{0.2cm}XX岁\hspace{0.2cm}中国}
\SimpleEntry{tel}
\SimpleEntry{e-mail}
\SimpleEntry{\href{https://kleinyang.com/}{https://kleinyang.com/}}

%************照片**************
%照片需要放到images文件夹下,名字必须是you.jpg,注意.jpg后缀(可以去resume.cls第101行处修改),如果不需要照片可以不添加此行命令
%0.15的意思是,照片的宽度是页面宽度的0.15倍,调整大小,避免遮挡文字
\yourphoto{0.13}

%***********个人信息**************
%\section{个人信息}
%
% \quad 性别:男 \hspace{2.35cm} 籍贯:  \hspace{2.25cm} 民族:汉族
%
% \quad 出生年月:  \hspace{0.5cm}政治面貌:  \hspace{0.5cm} 联系地址:


%***********教育背景**************
\section{教育背景}
%***第一个大括号里的内容向左对齐,第二个大括号里的内容向右对齐
%***\textbf{}括号里的字是粗体,\textit{}括号里的字是斜体
\datedsubsection{\textbf{MIT},计算机,GPA:XXX/100,综合排名:1/12138,\textit{本科}}{20XX.09 - 至今}

\begin{itemize}
	\item 国家奖学金、校优秀毕业生
\end{itemize}

%***********实习经历**************

\section{实习经历}

\datedsubsection{\textbf{Google}:SD}{20XX.0X - 20XX.0X}
\Content
{\textit{描述}:职位所需}
{\textit{我的职责}:干了什么}
{\textit{成果}:顺利完成实习任务、或者受到高度评价,收获了什么}



%***********过往经历**************
\section{项目竞赛经历}

\datedsubsection{\textbf{XXXX大学电子XXX研究所XXX项目}:XXX Dashboard}{20XX.0X - 20XX.0X}
\Content
{\textit{描述}:分析获取的XXX数据,在商定的范围内,建立一个研究模型,明确潜在变量和假设,得出XXX统计资料等相关因素对XXXX的可能影响,最后使用XXXX进行可视化,给予XXXXX支持。}
{\textit{我的职责}:使用XXXX进行可视化。}
{\textit{成果}:XXXX认可了我们的工作。}



\datedsubsection{\textbf{国家级大学生创新创业项目}:基于计算机视觉的XXXX\quad 负责人}{20XX.0X - 20XX.0X}
\Content
{\textit{描述}:以实现及时、准确、自动、智能地对XXXX,开发了基于区域推荐机制的XXXXX的算法模型以及基于XXXXXX的算法模型。}
{\textit{我的职责}:搭建XXXXX的算法模型,书写项目报告等。}
{\textit{成果}:获得了大学生创新创业项目一等奖、“挑战杯”XXXXX竞赛一等奖。}

%\datedsubsection{\textbf{美国大学生数学建模竞赛}:Vespa mandarinia invasion:Analysis and Strategy}{2021.01 - 2021.02}
%\Content
%{\textit{描述}:20XX年XX月,在XXXXX。由于XXXXX的潜在影响严重,根据任务需求,我们建立一系列模型来分析、讨论和评价关于XXXX的一些问题,最终建立了四个模型:模型I:XXX模型;模型II:XXX模型;模型III:XXX模型;模型IV:XXX模型。}
%{\textit{我的职责}:建立数学模型和算法模型,完成建模论文撰写。}
%{\textit{成果}:美国大学生数学建模竞赛X奖。}


\section{学生工作}
\datedsubsection{\textbf{推文撰写与微信公众号运营}}{20XX.0X - 20XX.0XXX}
\Content
{\textit{描述}:XXX}
{\textit{我的职责}:XXX}
{\textit{成果}:主稿的推文最高获得3W+的阅读量}



\section{专业技能}
\datedsubsection{\textbf{计算机方面}:Python、SQL、C++、PS、PR}{}
\datedsubsection{\textbf{英语方面}:CET-6}{}
%\datedsubsection{\textbf{专业证书}:}{}
\sepspace

\section{奖励荣誉}

\datedsubsection{\textbf{竞赛方面}:}{}
\datedsubsection{全国大学生XXXX一等奖}{20XX}
\datedsubsection{全国大学生数学建模大赛XXX}{20XX}
%\datedsubsection{\textbf{论文方面}:xxx}{2020}
%\datedsubsection{\textbf{奖学金方面}:}{}
%\datedsubsection{国家励志奖学金}{}
%\datedsubsection{校级人民奖学金}{X次}
\section{自我评价}
\datedsubsection{\quad \quad XXX}{}

\end{document}